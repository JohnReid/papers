\subsubsection*{Development of Inference Methods}

Edward uses class inheritance to provide a hierarchy of inference
methods. This enables fast experimentation on top of existing
algorithms, whether it be developing new black box algorithms or
new model-specific algorithms.

\begin{figure}[!htb]
\centering
\begin{tikzpicture}
    \begin{pgfonlayer}{nodelayer}
        \node [style=box] (0) at (2.25, 2) {Inference};
        \node [style=box] (1) at (-0.5, 0.5) {VariationalInference};
        \node [style=box] (2) at (5, 0.5) {MonteCarlo};
        \node [style=box] (3) at (-2.5, -1.25) {KLqp};
        \node [style=box] (4) at (-0.5, -1.25) {KLpq};
        \node [style=box] (5) at (1.5, -1.25) {MAP};
        \node [style=box] (6) at (1.5, -3) {Laplace};
        % \node [style=box] (7) at (4.0, -1.25) {GANInference};
        % \node [style=box] (8) at (4.0, -3) {WGANInference};
        \node [style=box] (9) at (3.5, -1.25) {MH};
        \node [style=box] (10) at (5, -1.25) {HMC};
        \node [style=box] (11) at (6.5, -1.25) {SGLD};
        % \node [style=box] (10) at (7.5, 0.5) {Exact};
        %\node [style=box] (10) at (-4, -1.25) {CollapsedInference};
        %\node [style=box] (11) at (-5, -1.25) {ConjugateInference};
    \end{pgfonlayer}
    \begin{pgfonlayer}{edgelayer}
        \draw [style=arrow, bend right] (0) to (1);
        \draw [style=arrow] (1) to (3);
        \draw [style=arrow] (1) to (4);
        \draw [style=arrow] (1) to (5);
        \draw [style=arrow] (5) to (6);
        % \draw [style=arrow] (1) to (7);
        % \draw [style=arrow] (7) to (8);
        \draw [style=arrow, bend left] (0) to (2);
        %\draw [style=arrow, bend left, draw=Gray] (0) to (2);
        \draw [style=arrow] (2) to (9);
        \draw [style=arrow] (2) to (10);
        \draw [style=arrow] (2) to (11);
        % \draw [style=arrow, bend left, out=17] (0) to (10);
        %\draw [style=arrow] (9) to (10);
        %\draw [style=arrow] (9) to (11);
    \end{pgfonlayer}
\end{tikzpicture}

\caption{Dependency graph of several inference methods. Nodes are classes in Edward
and arrows represent class inheritance.}
\end{figure}

There is a base class \texttt{Inference}, from which all inference
methods are derived from. Note that \texttt{Inference} says nothing
about the class of models that an algorithm must work with. One can
build inference algorithms which are tailored to a restricted class of
models available in Edward (such as differentiable models or
conditionally conjugate models), or even tailor it to a single model.
The algorithm can raise an error if the model is outside this class.

We organize inference under two paradigms:
\texttt{VariationalInference} and \texttt{MonteCarlo} (or more plainly,
optimization and sampling). These inherit from \texttt{Inference} and each
have their own default methods.

For example, developing a new variational inference algorithm is as simple as
inheriting from \texttt{VariationalInference} and writing a
\texttt{build_loss_and_gradients()} method. \texttt{VariationalInference} implements many default methods such
as \texttt{initialize()} with options for an optimizer.
For example, see the
importance weighted variational inference
script.%
\footnote{\url{https://github.com/blei-lab/edward/blob/master/examples/iwvi.py}}
