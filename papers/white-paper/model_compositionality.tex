\subsection*{Composing Random Variables}

Core to Edward's design is compositionality. Compositionality enables
fine control of modeling, where models are represented as a collection
of random variables.

We outline how to write popular classes of models using Edward:
directed graphical models, neural networks, Bayesian nonparametrics,
and probabilistic programs.

\subsubsection{Directed Graphical Models}

Graphical models are a rich formalism for specifying probability
distributions \citep{koller2009probabilistic}.
In Edward, directed edges in a graphical model are implicitly defined
when random variables are composed with one another. We illustrate
with a Beta-Bernoulli model,
\begin{equation*}
p(\mathbf{x}, \theta) =
\text{Beta}(\theta\mid 1, 1)
\prod_{n=1}^{50} \text{Bernoulli}(x_n\mid \theta),
\end{equation*}
where $\theta$ is a latent probability shared across the 50 data
points $\mathbf{x}\in\{0,1\}^{50}$.

\begin{lstlisting}[language=python]
from edward.models import Bernoulli, Beta

theta = Beta(a=1.0, b=1.0)
x = Bernoulli(p=tf.ones(50) * theta)
\end{lstlisting}

\begin{figure}[!htb]
\centering
\begin{tikzpicture}[x=1.7cm,y=1.8cm,scale=0.9]
%\begin{tikzpicture}[scale=0.6]

  % Nodes
  \node[latent] (theta) {$\theta$};
  \factor[right=of theta, xshift=0.3cm] {thetastar} {$\theta^*$} {} {};

  \factor[above=of thetastar] {n} {\texttt{tf.ones(50)}} {} {};
  \node[latent, right=of thetastar, xshift=-0.5cm] (x) {$\mbx$};
  \factor[right=of x, xshift=0.3cm] {xstar} {$\mbx^*$} {} {};

  % Edges
  \edge{theta}{thetastar};
  \edge{thetastar}{x};
  \edge{n}{x};
  \edge{x}{xstar};

\end{tikzpicture}

\caption{Computational graph for a Beta-Bernoulli program.}
\end{figure}

The random variable \texttt{x} ($\mathbf{x}$) is 50-dimensional,
parameterized by the random tensor $\theta^*$. Fetching the object
\texttt{x.value()} ($\mathbf{x}^*$) from session runs the graph: it simulates from
the generative process and outputs a binary vector of $50$ elements.

With computational graphs, it is also natural to build mutable states
within the probabilistic program. As a typical use of computational
graphs, such states can define model parameters, that is, parameters
that we will always compute point estimates for and not be uncertain
about. In TensorFlow, this is given by a \texttt{tf.Variable}.

\begin{lstlisting}[language=python]
from edward.models import Bernoulli

theta = tf.Variable(0.0)
x = Bernoulli(p=tf.ones(50) * tf.sigmoid(theta))
\end{lstlisting}

Another use case of mutable states is for building discriminative
models $p(\mathbf{y}\mid\mathbf{x})$, where $\mathbf{x}$ are features
that are input as training or test data. The program can be written
independent of the data, using a mutable state
(\texttt{tf.placeholder}) for $\mathbf{x}$ in its graph. During
training and testing, we feed the placeholder the appropriate values.

\subsubsection{Neural Networks}

As Edward uses TensorFlow, it is easy to construct neural networks for
probabilistic modeling \citep{rumelhart1988parallel}.
For example, one can specify stochastic neural networks
\citep{neal1990learning}.

High-level libraries such as
{Keras}\footnote{\url{http://keras.io}} and
{TensorFlow Slim}\footnote{\url{https://github.com/tensorflow/tensorflow/tree/master/tensorflow/contrib/slim}}
can be used to easily construct deep neural networks.
We illustrate this with a deep generative model over binary data
$\{\mathbf{x}_n\}\in\{0,1\}^{N\times 28*28}$.

\begin{figure}[!htb]
\centering
\begin{tikzpicture}

  % Nodes
  \node[latent] (z) {$\mbz_n$};
  \node[obs, right=of z] (x) {$\mbx_n$};

  \factor[empty, right=of z] {h} {} {} {};
  \factor[right=of z, yshift=1.0cm] {theta} {$\theta$} {} {};
  % \factor[left=of h, xshift=-0.5cm] {phi} {$\phi$} {} {};

  % Edges
  \edge{z}{x};
  % \draw[style=arrow, densely dotted, bend left] (x) to (z);
  \edge{theta}{x};
  % \draw[style=arrow, densely dotted] (phi) to (z);

  % Plates
  \plate[inner sep=0.25cm, yshift=0.05cm,
    label={[xshift=-14pt,yshift=14pt]south east:$N$}] {plate1} {
    (z)(x)
  } {};

\end{tikzpicture}

\caption{Graphical representation of a deep generative model.}
\end{figure}

The model specifies a generative process where for each
$n=1,\ldots,N$,
%
\begin{align*}
\mathbf{z}_n &\sim \text{Normal}(\mathbf{z}_n \mid \mathbf{0}, \mathbf{I}), \\
\mathbf{x}_n\mid \mathbf{z}_n &\sim \text{Bernoulli}(\mathbf{x}_n\mid
p=\mathrm{NN}(\mathbf{z}_n; \mathbf{\theta})).
\end{align*}
%
The latent space is $\mathbf{z}_n\in\mathbb{R}^d$ and the
likelihood is parameterized by a neural network $\mathrm{NN}$ with
parameters $\theta$. We will use a two-layer neural network with a
fully connected hidden layer of 256 units (with ReLU activation) and
whose output is $28*28$-dimensional. The output will be unconstrained,
parameterizing the logits of the Bernoulli likelihood.

With TensorFlow Slim, we write this model as follows:

\begin{lstlisting}[language=python]
from edward.models import Bernoulli, Normal
from tensorflow.contrib import slim

z = Normal(mu=tf.zeros([N, d]), sigma=tf.ones([N, d]))
h = slim.fully_connected(z, 256)
x = Bernoulli(logits=slim.fully_connected(h, 28 * 28, activation_fn=None))
\end{lstlisting}

With Keras, we write this model as follows:

\begin{lstlisting}[language=python]
from edward.models import Bernoulli, Normal
from keras.layers import Dense

z = Normal(mu=tf.zeros([N, d]), sigma=tf.ones([N, d]))
h = Dense(256, activation='relu')(z.value())
x = Bernoulli(logits=Dense(28 * 28)(h))
\end{lstlisting}

Keras and TensorFlow Slim automatically manage TensorFlow variables, which
serve as parameters of the high-level neural network layers. This
saves the trouble of having to manage them manually. However, note
that neural network parameters defined this way always serve as model
parameters. That is, the parameters are not exposed to the user so we
cannot be Bayesian about them with prior distributions.

\subsubsection{Bayesian Nonparametrics}

Bayesian nonparametrics enable rich probability models by working over
an infinite-dimensional parameter space \citep{hjort2010bayesian}.
Edward supports the two typical approaches to handling these models:
collapsing the infinite-dimensional space and lazily defining the
infinite-dimensional space.

For the collapsed approach, see the
Gaussian process classification
tutorial as an example. We specify distributions over the function
evaluations of the Gaussian process, and the Gaussian process is
implicitly marginalized out. This approach is also useful for Poisson
process models.

To work directly on the infinite-dimensional space, one can leverage
random variables with
control flow operations
in TensorFlow. At runtime, the control flow will lazily define any
parameters in the space necessary in order to generate samples. As an
example, we use a while loop to define a
Dirichlet process according to its stick breaking representation.

\subsubsection{Probabilistic Programs}

Probabilistic programs greatly expand the scope of probabilistic
models \citep{goodman2012church}.
Formally, Edward is a Turing-complete probabilistic programming
language. This means that Edward can represent any computable
probability distribution.

\begin{figure}[!htb]
\centering
\begin{tikzpicture}[x=1.7cm,y=1.8cm,scale=0.9]
%\begin{tikzpicture}[scale=0.6]

  % Nodes
  \node[latent] (p) {$\mbp$};
  \factor[right=of p, xshift=0.3cm] {pstar} {$\mbp^*$} {} {};

  \factor[above=of pstar] {n} {} {} {};
  \factor[empty, right=of n, yshift=0.1cm] {nn} {\texttt{tf.while_loop(...)}} {} {};
  \factor[left=of n, xshift=-0.5cm] {astar} {$\mba^*$} {} {};
  \node[latent, left=of astar, xshift=0.5cm] (a) {$\mba$};
  \node[latent, right=of pstar, xshift=-0.5cm] (x) {$\mbx$};
  \factor[right=of x, xshift=0.3cm] {xstar} {$\mbx^*$} {} {};

  % Edges
  \edge{p}{pstar};
  \edge{pstar}{x};
  \edge{n}{x};
  \edge{x}{xstar};
  \edge{a}{astar};
  \edge{astar}{n};

\end{tikzpicture}

\caption{Computational graph for a probabilistic program with stochastic control flow.}
\end{figure}

Random variables can be composed with control flow operations,
enabling probabilistic programs with stochastic control flow.
%
Stochastic control flow defines dynamic conditional dependencies,
known in the literature as contingent or existential dependencies
\citep{mansinghka2014venture,wu2016swift}.
See above, where $\mathbf{x}$ may or may not depend on $\mathbf{a}$
for a given execution.

Stochastic control flow produces difficulties for algorithms that
leverage the graph structure; the relationship of conditional
dependencies changes across execution traces.
Importantly, the computational graph provides an elegant way of
teasing out static conditional dependence structure ($\mathbf{p}$)
from dynamic dependence structure ($\mathbf{a})$. We can perform
model parallelism over the static structure with GPUs and batch
training, and use generic computations to handle the dynamic
structure.
