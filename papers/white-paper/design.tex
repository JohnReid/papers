\section{Design}
\label{sec:design}

Edward's design reflects the building blocks for probabilistic
modeling. It defines interchangeable components, enabling rapid
experimentation and research with probabilistic models.

Edward is named after the innovative statistician
George Edward Pelham Box. Edward follows Box's philosophy of statistics and
machine learning \citep{box1976science}.

First gather data from some real-world phenomena. Then cycle through
Box's loop \citep{blei2014build}.

\begin{enumerate}
\item Build a probabilistic model of the phenomena.
\item Reason about the phenomena given model and data.
\item Criticize the model, revise and repeat.
\end{enumerate}

\begin{figure}[htb]
\centering
\begin{tikzpicture}
	\begin{pgfonlayer}{nodelayer}
		\node [style=box] (0) at (-2.5, 0) {Model};
		\node [style=box] (1) at (0, 0) {Infer};
		\node [style=box, fill=MidnightBlue, draw=white] (2) at (0, 1.5)
		{\color{white}Data};
		\node [style=box] (3) at (2.75, 0) {Criticize};
	\end{pgfonlayer}
	\begin{pgfonlayer}{edgelayer}
		\draw [style=arrow] (0) to (1);
		\draw [style=arrow] (2) to (1);
		\draw [style=arrow] (1) to (3);
		\draw [style=arrow, bend left=75, looseness=1.25] (3) to (0);
	\end{pgfonlayer}
\end{tikzpicture}

\caption{Box's loop.}
\end{figure}


Here's a toy example. A child flips a coin ten times, with the set of outcomes
being

\texttt{{[}0,\ 1,\ 0,\ 0,\ 0,\ 0,\ 0,\ 0,\ 0,\ 1{]}},

where \texttt{0}
denotes tails and \texttt{1} denotes heads. She is interested in the
probability that the coin lands heads. To analyze this, she first
builds a model: suppose she assumes the coin flips are independent and
land heads with the same probability. Second, she reasons about the
phenomenon: she infers the model's hidden structure given data.
Finally, she criticizes the model: she analyzes whether her model
captures the real-world phenomenon of coin flips. If it doesn't, then
she may revise the model and repeat.

We describe modules enabling this analysis.
