\section{Compositional Representations for Inference}
\label{sec:inference}

We described random variables as a representation for building
rich probabilistic programs over computational graphs.  We now
describe a compositional representation for inference.
We desire two criteria: (a) support for many classes of
inference, where the form of the inferred posterior depends on the
algorithm; and (b) invariance of inference under the computational
graph, that is, the posterior can be further composed as part of
another model.

To explain our approach, we will use a simple hierarchical model as a
running example. \Cref{fig:hierarchical_model_example} displays a
joint distribution $p(\mbx, \mbz, \beta)$ of data $\mbx$, local
variables $\mbz$, and global variables $\beta$. The ideas here extend
to more expressive programs.

\begin{figure}[!h]
\begin{subfigure}{0.35\columnwidth}
  \centering
  \begin{tikzpicture}

  % Nodes
  \node[latent]               (beta)      {$\beta$} ;
  \node[left=0.4cm of beta]               (betaph)      {} ;
  \node[latent, below=1.0cm of betaph] (z)    {$\mbz_n$} ;
  \node[obs, right=1.0cm of z]      (x)        {$\mbx_n$} ;

  % Edges
  % \edge{beta}{z};
  \edge{beta}{x};
  \edge{z}{x};

  % Plates
  \plate[inner sep=0.3cm,
    label={[xshift=-15pt,yshift=15pt]south east:$N$}] {plate1} {
    (x)(z)
  } {};

\end{tikzpicture}

\end{subfigure}%
\begin{subfigure}{0.6\columnwidth}
  \centering
\begin{lstlisting}[language=python]
N = 10000  # number of data points
D = 2  # data dimension
K = 5  # number of clusters

beta = Normal(mu=tf.zeros([K, D]), sigma=tf.ones([K, D]))
z = Categorical(logits=tf.zeros([N, K]))
x = Normal(mu=tf.gather(beta, z), sigma=tf.ones([N, D]))
\end{lstlisting}
\end{subfigure}
\caption{Hierarchical model: \textbf{(left)} graphical model; \textbf{(right)}
  probabilistic program. It is a mixture of Gaussians over
  $D$-dimensional data $\{x_n\}\in\mathbb{R}^{N\times D}$. There are
  $K$ latent cluster means $\beta\in\mathbb{R}^{K\times D}$.}
\label{fig:hierarchical_model_example}
\end{figure}

\subsection{Inference as Stochastic Graph Optimization}
\label{sub:inference}

The goal of inference is to
calculate the posterior distribution
$p(\mathbf{z}, \beta\mid \mathbf{x}_{\text{train}}; \mbtheta)$
given data $\mbx_{\text{train}}$,
where
$\mbtheta$ are any model parameters that we will compute point estimates
for.\footnote{%
For example, we could replace \texttt{x}'s \texttt{sigma}
argument with \texttt{tf.exp(tf.Variable(0.0))*tf.ones([N, D])}. This
defines a model parameter initialized at 0 and positive-constrained.}
We formalize this as the following optimization problem:
\begin{align}
  \label{eq:inference-optimization}
\min_{\mblambda,\mbtheta}
\mathcal{L}(
p(\mathbf{z}, \beta\mid \mathbf{x}_{\text{train}}; \mbtheta),~
q(\mathbf{z}, \beta; \mblambda)
),
\end{align}
where $q(\mathbf{z}, \beta; \mblambda)$ is an approximation to the
posterior $p(\mathbf{z}, \beta\g \mbx_{\text{train}};\mbtheta)$, and
$\mathcal{L}$ is a loss function with respect to $p$ and $q$.

The choice of approximation $q$, loss $\mathcal{L}$, and rules to update
parameters $\{\mbtheta,\mblambda\}$ are specified by an inference algorithm.
(Note $q$ can be nonparametric, such as a point or a collection of
samples.)

In Edward, we write this problem as follows:

\begin{lstlisting}[language=python]
inference = ed.Inference({beta: qbeta, z: qz}, data={x: x_train})
\end{lstlisting}

\texttt{Inference} is an abstract class which takes two inputs.  The
first is a collection of latent random variables \texttt{beta} and
\texttt{z}, associated to their ``posterior variables'' \texttt{qbeta} and
\texttt{qz} respectively. The second is a collection of observed random variables
\texttt{x}, which is associated to their realizations \texttt{x_train}.

The idea is that \texttt{Inference} defines and
solves the optimization in \Cref{eq:inference-optimization}. It
adjusts parameters of the distribution of \texttt{qbeta}
and \texttt{qz} (and any model parameters) to be close to the
posterior.

Class methods are available to finely control the inference. Calling
\texttt{inference.initialize()} builds a computational graph to update
$\{\mbtheta,\mblambda\}$. Calling \texttt{inference.update()} runs
this computation once to update $\{\mbtheta,\mblambda\}$; we call the
method in a loop until convergence. Importantly, no efficiency is lost
in Edward's language: the computational graph is the same as if it
were handwritten for a
specific model. This means the runtime is the same; also see our
experiments in \Cref{sub:gpu}.

A key concept in Edward is that there is no distinct ``model''
or ``inference'' block. A model is simply a collection of random
variables, and inference is a way of modifying parameters in that
collection subject to another. This reductionism offers
significant flexibility. For example, we can
infer only parts of a model (e.g.,
layer-wise training \citep{hinton2006fast}),
infer parts used in multiple models
(e.g., multi-task learning), or
plug in a posterior into a new model
(e.g., Bayesian updating).

\subsection{Classes of Inference}

The design of \texttt{Inference} is very general.  We describe
subclasses to represent many algorithms below: variational inference,
Monte Carlo, and \acrlongpl{GAN}.

Variational inference posits a family of approximating distributions
and finds the closest member in the family to the posterior
\citep{jordan1999introduction}.  In Edward, we build the variational
family in the graph; see \Cref{fig:inference} (left). For our running
example, the
family has mutable variables as parameters
$\mblambda=\{\pi,\mu,\sigma\}$, where
$q(\beta;\mu,\sigma) = \operatorname{Normal}(\beta; \mu,\sigma)$ and
$q(\mbz;\pi) = \operatorname{Categorical}(\mbz;\pi)$.

\begin{figure}[!h]
\begin{subfigure}{0.5\columnwidth}
  \centering
\begin{lstlisting}[language=Python]
qbeta = Normal(
  mu=tf.Variable(tf.zeros([K, D])),
  sigma=tf.exp(tf.Variable(tf.zeros([K, D]))))
qz = Categorical(
  logits=tf.Variable(tf.zeros([N, K])))

inference = ed.VariationalInference(
  {beta: qbeta, z: qz}, data={x: x_train})
\end{lstlisting}
\end{subfigure}%
\begin{subfigure}{0.5\columnwidth}
  \centering
\begin{lstlisting}[language=Python]
T = 10000  # number of samples
qbeta = Empirical(
  params=tf.Variable(tf.zeros([T, K, D])))
qz = Empirical(
  params=tf.Variable(tf.zeros([T, N])))

inference = ed.MonteCarlo(
  {beta: qbeta, z: qz}, data={x: x_train})
\end{lstlisting}
\end{subfigure}
\caption{\textbf{(left)} Variational inference. \textbf{(right)} Monte Carlo.}
\label{fig:inference}
\end{figure}

Specific variational algorithms inherit from the
\texttt{VariationalInference} class.  Each defines its own methods,
such as a loss function and gradient.
For example, we represent
\gls{MAP}
estimation with an approximating family (\texttt{qbeta} and
\texttt{qz}) of \texttt{PointMass} random variables, i.e., with all
probability mass concentrated at a point.
\texttt{MAP} inherits from \texttt{VariationalInference} and defines
the negative log joint density as the loss function; it uses existing optimizers inside
TensorFlow.
In \Cref{sub:recent}, we experiment with multiple gradient estimators
for black box variational inference \citep{ranganath:2014}. Each
estimator implements the same loss (an objective proportional to
the divergence $\operatorname{KL}(q\gg p)$) and a different update rule
(stochastic gradient).

Monte Carlo approximates the posterior using samples
\citep{robert1999monte}. Monte Carlo is an inference where the
approximating family is an empirical distribution,
$q(\beta; \{\beta^{(t)}\}) = \frac{1}{T}\sum_{t=1}^T \delta(\beta,
\beta^{(t)})$ and
$q(\mbz; \{\mbz^{(t)}\}) = \frac{1}{T}\sum_{t=1}^T \delta(\mbz,
\mbz^{(t)})$. The parameters are
$\mblambda=\{\beta^{(t)},\mbz^{(t)}\}$.  See \Cref{fig:inference}
(right).
Monte Carlo algorithms proceed by updating one sample
$\beta^{(t)},\mbz^{(t)}$ at a time in the empirical approximation.
Specific \glsunset{MC}\gls{MC} samplers determine the update rules:
they can use gradients such as in Hamiltonian Monte Carlo
\citep{neal2011mcmc} and graph
structure such as in sequential Monte Carlo \citep{doucet2001introduction}.



\begin{figure}[tb]
\begin{subfigure}{0.4\columnwidth}
  \centering
  \begin{tikzpicture}

  % Nodes
  \node[latent] (eps0) {$\mbepsilon_n$};
  \factor[minimum size=0.3cm, below=1.2cm of eps0] {x} {} {} {};

  \factor[empty, below=of eps0] {h} {} {} {};
  \factor[right=of h, xshift=0.7cm] {theta} {$\theta$} {} {};

  \node[right=of x, xshift=-0.9cm] (xlabel) {$\mbx_n$};

  % Edges
  \edge{eps0}{x};
  \edge{theta}{x};

  % Plates
  \plate[inner sep=0.4cm, yshift=0.05cm,xshift=0.15cm,
    label={[xshift=-14pt,yshift=14pt]south east:$N$}] {plate1} {
    (eps0)(x)
  } {};

\end{tikzpicture}

\end{subfigure}%
\begin{subfigure}{0.6\columnwidth}
  \centering
\begin{lstlisting}[language=python]
def generative_network(eps):
  h = Dense(256, activation='relu')(eps)
  return Dense(28 * 28, activation=None)(h)

def discriminative_network(x):
  h = Dense(28 * 28, activation='relu')(x)
  return Dense(h, activation=None)(1)

# Probabilistic model
eps = Normal(mu=tf.zeros([N, d]), sigma=tf.ones([N, d]))
x = generative_network(eps)

inference = ed.GANInference(data={x: x_train},
    discriminator=discriminative_network)
\end{lstlisting}
\end{subfigure}
\caption{\Acrlongpl{GAN}:
\textbf{(left)} graphical model; \textbf{(right)} probabilistic program.
The model (generator) uses a parameterized function (discriminator)
for training.
}
\label{fig:gan}
\end{figure}

Edward also supports non-Bayesian methods such as \glspl{GAN}
\citep{goodfellow2014generative}.
See \Cref{fig:gan}.
The model posits
random noise \texttt{eps} over $N$ data points, each with $d$
dimensions; this random noise feeds into a
\texttt{generative_network} function, a neural network that outputs
real-valued data \texttt{x}.
In addition, there is a \texttt{discriminative_network}
which takes data as input and outputs the probability that
the data is real (in logit parameterization). We build
\texttt{GANInference}; running it optimizes parameters inside the two
neural network functions. This approach extends to many advances in
\glspl{GAN} (e.g., \citet{denton2015deep,li2015generative}).

Finally, one can design algorithms that would otherwise require tedious
algebraic manipulation. With symbolic algebra on nodes of the
computational graph, we can uncover conjugacy relationships between
random variables. Users can then integrate out variables to
automatically derive classical Gibbs \citep{gelfand1990sampling},
mean-field updates \citep{bishop2006pattern}, and exact inference.
These algorithms are being currently developed in Edward.

\subsection{Composing Inferences}

Core to Edward's design is that inference can be written as a collection
of separate inference programs. Below we demonstrate variational EM,
with an (approximate) E-step over local variables and an M-step over
global variables. We instantiate two algorithms, each of which
conditions on inferences from the other, and
we alternate with one update of each \citep{neal1993new},
\begin{lstlisting}[language=Python]
qbeta = PointMass(params=tf.Variable(tf.zeros([K, D])))
qz = Categorical(logits=tf.Variable(tf.zeros([N, K])))

inference_e = ed.VariationalInference({z: qz}, data={x: x_train, beta: qbeta})
inference_m = ed.MAP({beta: qbeta}, data={x: x_train, z: qz})
...
for _ in range(10000):
  inference_e.update()
  inference_m.update()
\end{lstlisting}
This extends to many other cases such as
exact EM for exponential families,
contrastive divergence \citep{hinton2002training},
pseudo-marginal methods \citep{andrieu2009pseudo},
and Gibbs sampling within variational inference
\citep{wang2012truncation,Hoffman:2015}.
We can also write message passing algorithms, which solve a collection
of local inference problems \citep{koller2009probabilistic}.  For
example, classical message passing uses exact local inference and
expectation propagation locally minimizes the Kullback-Leibler
divergence, $\text{KL}(p\gg q)$
\citep{minka2001expectation}.

\subsection{Data Subsampling}
\label{sub:batch_training}

Stochastic optimization \citep{bottou2010large} scales inference to
massive data and is key to
algorithms such as stochastic gradient Langevin dynamics
\citep{welling2011bayesian} and stochastic variational inference
\citep{hoffman2013stochastic}.  The idea is to cheaply estimate the
model's log joint density in an unbiased way.  At each step, one
subsamples a data set $\{x_m\}$ of size $M$ and then scales densities
with respect to local variables,
\begin{align*}
  \log p(\mbx, \mbz, \beta)
  & = \log p(\beta) +
  \sum_{n=1}^N\Big[\log p(x_n \g z_n, \beta) + \log p(z_n \g \beta)\Big]
  \\
  & \approx \log p(\beta) +
  \frac{N}{M}\sum_{m=1}^M\Big[\log p(x_m \g z_m, \beta) + \log p(z_m \g \beta)\Big].
\end{align*}
To support stochastic optimization, we represent only a subgraph of
the full model. This prevents reifying the full model, which can lead
to unreasonable memory consumption \citep{tristan2014augur}.  During
initialization, we pass in a dictionary to properly scale the
arguments. See \Cref{fig:hierachical_model_batch}.

\begin{figure}[!htb]
\begin{subfigure}{0.3\columnwidth}
  \centering
  \begin{tikzpicture}

  % Nodes
  \node[latent]               (beta)      {$\beta$} ;
  \node[left=0.4cm of beta]               (betaph)      {} ;
  \node[latent, below=1.0cm of betaph] (z)    {$\mbz_m$} ;
  \node[obs, right=1.0cm of z]      (x)        {$\mbx_m$} ;

  % Edges
  % \edge{beta}{z};
  \edge{beta}{x};
  \edge{z}{x};

  % Plates
  \plate[inner sep=0.3cm,
    label={[xshift=-15pt,yshift=15pt]south east:$M$}] {plate1} {
    (x)(z)
  } {};

\end{tikzpicture}

  \label{sub:hierachical_model_math}
\end{subfigure}%
\begin{subfigure}{0.6\columnwidth}
  \centering
\begin{lstlisting}[language=python]
beta = Normal(mu=tf.zeros([K, D]), sigma=tf.ones([K, D]))
z = Categorical(logits=tf.zeros([M, K]))
x = Normal(mu=tf.gather(beta, z), sigma=tf.ones([M, D]))

qbeta = Normal(mu=tf.Variable(tf.zeros([K, D])),
               sigma=tf.nn.softplus(tf.Variable(tf.zeros([K, D]))))
qz = Categorical(logits=tf.Variable(tf.zeros([M, D])))

inference = ed.VariationalInference({beta: qbeta, z: qz}, data={x: x_batch})
inference.initialize(scale={x: float(N)/M, z: float(N)/M})
\end{lstlisting}
  \label{sub:hierarchical_model_code}
\end{subfigure}
\caption{Data subsampling with a hierarchical model. We define a
subgraph of the full model, forming a plate of size $M$
rather than $N$. We then scale all local random variables by $N/M$.}
\label{fig:hierachical_model_batch}
\end{figure}

Conceptually, the scale argument represents scaling for each random
variable's plate, as if we had seen that random variable $N / M$ as
many times.  As an example, \Cref{appendix:svi} shows how to implement
stochastic variational inference in Edward.
The approach extends naturally to streaming data
\citep{doucet2000on,broderick2013streaming,mcinerney2015population},
dynamic batch sizes, and data structures in which working on a
subgraph does not immediately apply
\citep{binder1997space,johnson2014stochastic,foti2014stochastic}.

