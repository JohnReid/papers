\vspace{-0.5ex}
\section{Compositional Representations for Probabilistic Models}
\label{sec:modeling_language}
\vspace{-0.5ex}

We first develop compositional representations for probabilistic
models. We desire two criteria: (a) integration with computational
graphs, an efficient
framework where nodes represent operations on data and edges
represent data communicated between them
\citep{culler1986dataflow}; and (b)
invariance of the representation under the graph, that is, the
representation can be
reused during inference.

Edward defines random variables as the key compositional representation.
They are class objects with methods, for example, to compute the log
density and to sample. Further, each random variable $\mbx$ is
associated to a tensor (multi-dimensional array) $\mbx^*$, which
represents a single sample $\mbx^*\sim p(\mbx)$. This association
embeds the random variable onto a computational graph on tensors.

The design's simplicity makes it easy to develop probabilistic programs in a
computational graph framework. Importantly, all computation is
represented on the graph. This enables one to compose random
variables with complex deterministic structure such as deep neural
networks, a diverse set of math operations, and third party libraries
that build on the same framework. The design also enables compositions
of random variables to capture complex stochastic structure.

As an illustration, we use a Beta-Bernoulli model,
$p(\mbx, \theta) = \operatorname{Beta}(\theta\g 1, 1) \prod_{n=1}^{50}
\operatorname{Bernoulli}(x_n\g \theta)$, where $\theta$ is a latent
probability shared across the 50 data points $\mbx\in\{0,1\}^{50}$.
The random variable \texttt{x} is 50-dimensional, parameterized by the
random tensor $\theta^*$. Fetching the object \texttt{x} runs the
graph: it simulates from the generative process and outputs a binary
vector of $50$ elements.

\begin{figure}[!htb]
\begin{subfigure}{0.3\columnwidth}
  \centering
\begin{lstlisting}[language=python]
theta = Beta(a=1.0, b=1.0)
x = Bernoulli(p=tf.ones(50) * theta)
\end{lstlisting}
\end{subfigure}%
\begin{subfigure}{0.65\columnwidth}
  \centering
  \begin{tikzpicture}[x=1.7cm,y=1.8cm,scale=0.9]
%\begin{tikzpicture}[scale=0.6]

  % Nodes
  \node[latent] (p) {$p$};
  \factor[right=of p, xshift=0.3cm] {thetastar} {$p^*$} {} {};

  \factor[above=of thetastar] {n} {\texttt{tf.ones(10)}} {} {};
  \node[latent, right=of thetastar, xshift=-0.5cm] (x) {$\mbx$};
  \factor[right=of x, xshift=0.3cm] {xstar} {$\mbx^*$} {} {};

  % Edges
  \edge{p}{thetastar};
  \edge{thetastar}{x};
  \edge{n}{x};
  \edge{x}{xstar};

\end{tikzpicture}

\end{subfigure}
\caption{Beta-Bernoulli program \textbf{(left)} alongside its
computational graph \textbf{(right)}.
Fetching $\mbx$ from the graph generates a binary vector of $50$ elements.
}
\label{fig:beta_bernoulli}
\end{figure}

All computation is registered symbolically on random variables and not
over their execution. Symbolic representations
do not require reifying the full model, which leads to unreasonable
memory consumption for large models \citep{tristan2014augur}.
Moreover, it enables us to simplify both deterministic and stochastic
operations in the graph, before executing any code
\citep{scibior2015practical,zinkov2016composing}.

With computational graphs, it is also natural to build mutable states
within the probabilistic program. As a typical use of computational
graphs, such states can define model parameters; in TensorFlow, this
is given by a \texttt{tf.Variable}. Another use case is for building
discriminative models $p(\mby\g\mbx)$, where $\mbx$ are features that
are input as training or test data. The program can be written
independent of the data, using a mutable state
(\texttt{tf.placeholder}) for $\mbx$ in its graph. During training
and testing, we feed the placeholder the appropriate values.

In \Cref{appendix:model}, we provide examples of a Bayesian neural
network for classification (\ref{appendix:bnn}), latent Dirichlet
allocation (\ref{appendix:lda}), and Gaussian matrix factorization
(\ref{appendix:gaussian_mf}). We present others below.

\subsection{Example: Variational Auto-encoder}
\label{sub:vae}

\begin{figure}[tb]
\begin{subfigure}{0.225\columnwidth}
  \centering
  \begin{tikzpicture}

  % Nodes
  \node[latent] (z) {$\mbz_n$};
  \node[obs, below=of z] (x) {$\mbx_n$};

  \factor[empty, below=of z] {h} {} {} {};
  \factor[right=of h, xshift=0.5cm] {theta} {$\theta$} {} {};
  \factor[left=of h, xshift=-0.5cm] {phi} {$\phi$} {} {};

  % Edges
  \edge{z}{x};
  \draw[style=arrow, densely dotted, bend left] (x) to (z);
  \edge{theta}{x};
  \draw[style=arrow, densely dotted] (phi) to (z);

  % Plates
  \plate[inner sep=0.4cm, yshift=0.05cm,
    label={[xshift=-14pt,yshift=14pt]south east:$N$}] {plate1} {
    (z)(x)
  } {};

\end{tikzpicture}

  \label{sub:vae_math}
\end{subfigure}%
\begin{subfigure}{0.65\columnwidth}
  \centering
\begin{lstlisting}[language=python]
# Probabilistic model
z = Normal(mu=tf.zeros([N, d]), sigma=tf.ones([N, d]))
h = Dense(256, activation='relu')(z)
x = Bernoulli(logits=Dense(28 * 28, activation=None)(h))

# Variational model
qx = tf.placeholder(tf.float32, [N, 28 * 28])
qh = Dense(256, activation='relu')(qx)
qz = Normal(mu=Dense(d, activation=None)(qh),
            sigma=Dense(d, activation='softplus')(qh))
\end{lstlisting}
  \label{sub:vae_code}
\end{subfigure}
\caption{\Acrlong{VAE} for a data set of $28\times 28$ pixel images:
\textbf{(left)} graphical model, with dotted lines for the inference
model; \textbf{(right)} probabilistic program,
with 2-layer neural networks.
}
\label{fig:vae}
\end{figure}

\Cref{fig:vae} implements a \gls{VAE}
\citep{kingma2014autoencoding,rezende2014stochastic} in Edward. It
comprises a probabilistic model over data and a variational model
designed to approximate the former's posterior.
Here we use random variables to construct both
the probabilistic model and the variational model; they are fit
during inference (more details in \Cref{sec:inference}).

There are $N$ data points $x_n\in\{0,1\}^{28\cdot 28}$ each with
$d$ latent variables, $z_n\in\mathbb{R}^d$. The program uses Keras
\citep{chollet2015keras} to define neural networks. The
probabilistic model is parameterized by a 2-layer neural network, with
256 hidden units (and ReLU activation), and generates
$28\times 28$ pixel images. The variational model is parameterized by
a 2-layer inference network, with 256 hidden units and
outputs parameters of a normal posterior approximation.

The probabilistic program is concise. Core elements of the
\gls{VAE}---such as its distributional assumptions and neural net
architectures---are all extensible. With model compositionality, we
can embed it into more complicated models
\citep{gregor2015draw,rezende2016one} and for other learning tasks
\citep{kingma2014semi}. With inference compositionality (which we
discuss in \Cref{sec:inference}), we can embed it into more complicated algorithms, such
as with expressive variational approximations
\citep{rezende2015variational,tran2016variational,kingma2016improving}
and alternative objectives
\citep{ranganath2016operator,li2016variational,dieng2016chi}.

\subsection{\hspace{-0.225em}Example: Bayesian Recurrent Neural Network with Variable Length}

Random variables can also be composed with control flow operations.
As an example, \Cref{fig:bayesian_rnn} implements a Bayesian \glsreset{RNN}\gls{RNN} with
variable length.
The data is a sequence of inputs $\{\mbx_1,\ldots,\mbx_T\}$ and
outputs $\{y_1,\ldots,y_T\}$ of length $T$ with
$\mbx_t\in\mathbb{R}^{D}$ and $y_t\in\mathbb{R}$ per time step.
For $t=1,\ldots,T$,
a \gls{RNN} applies the update
\begin{equation*}
  \mbh_t = \operatorname{tanh}(\mbW_h \mbh_{t-1} + \mbW_x \mbx_t + \mbb_h),
\end{equation*}
where the previous hidden state is
$\mbh_{t-1}\in\mathbb{R}^H$.
We feed each hidden state into the output's likelihood,
$y_t \sim \operatorname{Normal}(\mbW_y \mbh_t + \mbb_y, 1)$, and
we place a standard normal prior over all parameters
$\{\mbW_h\in\mathbb{R}^{H\times H}, \mbW_x\in\mathbb{R}^{D\times H},
\mbW_y\in\mathbb{R}^{H\times 1},
\mbb_h\in\mathbb{R}^H,\mbb_y\in\mathbb{R}\}$. Our implementation is
dynamic: it differs from a \gls{RNN} with fixed length, which
pads and unrolls the computation.

\begin{figure}[tb]
\begin{subfigure}{0.35\columnwidth}
  \centering
  \begin{tikzpicture}

  % Nodes
  \node[empty]              (dot)      {} ;
  \node[obs, right=of dot, xshift=-0.75cm]        (xt) {$\mbx_t$} ;
  \node[latent, left=of xt, xshift=0.75cm, yshift=-0.75cm]  (bh)  {$\mbb_h$} ;
  \node[latent, above=of bh, yshift=-0.9cm]      (Wx) {$\mbW_x$} ;
  \node[latent, above=of Wx, yshift=-0.9cm]      (Wh)      {$\mbW_h$} ;
  \node[latent, below=of bh, yshift=0.9cm]      (Wy)      {$\mbW_y$} ;
  \node[latent, below=of Wy, yshift=1.0cm]      (by)      {$\mbb_y$} ;

  \factor[below=0.75cm of xt] {ht} {} {} {};
  \node[right=0.03cm of ht, yshift=-0.3cm] (htteyt) {$\mbh_t$};
  \node[empty, left=of ht]      (htminus)    {$\cdots$} ;
  \node[empty, right=of ht]      (htplus)    {$\cdots$} ;

  \node[obs, below=0.75cm of ht]      (yt)    {$\mby_t$} ;
  \node[empty, left=of yt]      (ytminus)    {} ;
  \node[empty, right=of yt]      (ytplus)    {} ;

  % Edges
  \edge{Wh}{ht};
  \edge{Wx}{ht};
  \edge{bh}{ht};
  \edge{xt}{ht};
  \edge{Wy}{yt};
  \edge{by}{yt};
  \edge{ht}{yt};
  \edge{htminus}{ht};
  \edge{ht}{htplus};

\end{tikzpicture}

\end{subfigure}%
\begin{subfigure}{0.6\columnwidth}
\begin{lstlisting}[language=python]
def rnn_cell(hprev, xt):
  return tf.tanh(tf.dot(hprev, Wh) + tf.dot(xt, Wx) + bh)

Wh = Normal(mu=tf.zeros([H, H]), sigma=tf.ones([H, H]))
Wx = Normal(mu=tf.zeros([D, H]), sigma=tf.ones([D, H]))
Wy = Normal(mu=tf.zeros([H, 1]), sigma=tf.ones([H, 1]))
bh = Normal(mu=tf.zeros(H), sigma=tf.ones(H))
by = Normal(mu=tf.zeros(1), sigma=tf.ones(1))

x = tf.placeholder(tf.float32, [None, D])
h = tf.scan(rnn_cell, x, initializer=tf.zeros(H))
y = Normal(mu=tf.matmul(h, Wy) + by, sigma=1.0)
\end{lstlisting}
\end{subfigure}
\caption{Bayesian \glsunset{RNN}\gls{RNN}: \textbf{(left)} graphical model;
  \textbf{(right)} probabilistic program. The program has an unspecified number
  of time steps; it uses a symbolic for loop (\texttt{tf.scan}). }
\label{fig:bayesian_rnn}
\end{figure}

\subsection{Stochastic Control Flow and Model Parallelism}

\begin{figure}[!htb]
  \centering
  \begin{tikzpicture}[x=1.7cm,y=1.8cm,scale=0.9]
%\begin{tikzpicture}[scale=0.6]

  % Nodes
  \node[latent] (p) {$\mbp$};
  \factor[right=of p, xshift=0.3cm] {pstar} {$\mbp^*$} {} {};

  \factor[above=of pstar] {n} {} {} {};
  \factor[empty, right=of n, yshift=0.1cm] {nn} {\texttt{tf.while_loop(...)}} {} {};
  \factor[left=of n, xshift=-0.5cm] {astar} {$\mba^*$} {} {};
  \node[latent, left=of astar, xshift=0.5cm] (a) {$\mba$};
  \node[latent, right=of pstar, xshift=-0.5cm] (x) {$\mbx$};
  \factor[right=of x, xshift=0.3cm] {xstar} {$\mbx^*$} {} {};

  % Edges
  \edge{p}{pstar};
  \edge{pstar}{x};
  \edge{n}{x};
  \edge{x}{xstar};
  \edge{a}{astar};
  \edge{astar}{n};

\end{tikzpicture}

\caption{Computational graph for a probabilistic program with stochastic control flow.
}
\label{fig:dynamic}
\end{figure}

Random variables can also be placed in the control flow itself,
enabling probabilistic programs with stochastic control flow.
Stochastic control flow defines dynamic conditional dependencies,
known in the literature as contingent or existential dependencies
\citep{mansinghka2014venture,wu2016swift}. See \Cref{fig:dynamic},
where $\mbx$ may or may not depend on $\mba$ for a given execution.
In \Cref{appendix:dirichlet_process}, we use stochastic control flow
to implement a Dirichlet process mixture model.
Tensors with stochastic shape are also possible: for
example, \texttt{tf.zeros(Poisson(lam=5.0))} defines a vector of zeros
with length given by a Poisson draw with rate $5.0$.

Stochastic control flow produces difficulties for algorithms that use
the graph structure because the relationship of conditional
dependencies changes across execution traces. The computational
graph, however, provides an elegant way of teasing out static
conditional dependence structure ($\mbp$) from dynamic dependence
structure ($\mba)$. We can perform model parallelism (parallel
computation across components of the model) over the static structure
with \glspl{GPU} and batch training. We can use more generic
computations to handle the dynamic structure.

