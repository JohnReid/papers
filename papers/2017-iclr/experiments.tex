\section{Experiments}
\label{sec:experiments}

In this section, we illustrate two main benefits of Edward:
flexibility and efficiency. For the former, we
show how it is easy to compare different inference algorithms on the
same model. For the latter, we show how it is easy to get significant
speedups by exploiting computational graphs.

\subsection{Recent Methods in Variational Inference}
\label{sub:recent}

\begin{table}[tb]
\centering
\begin{tabular}{lcc}
\toprule
Inference method & Negative log-likelihood
\\
\midrule
\gls{VAE} \citep{kingma2014autoencoding} & $\le$ 88.2 \\
\gls{VAE} without analytic KL & $\le$ 89.4 \\
\gls{VAE} with analytic entropy & $\le$ 88.1 \\
\gls{VAE} with score function gradient & $\le$ 87.9 \\
Normalizing flows \citep{rezende2015variational} & $\le$ 85.8 \\
Hierarchical variational model \citep{ranganath2016hierarchical} & $\le$ 85.4 \\
Importance-weighted auto-encoders ($K=50$) \citep{burda2016importance}
& $\le$ 86.3 \\
\acrshort{HVM} with \acrshort{IWAE} objective ($K=5$)
& $\le$ 85.2 \\
R\'{e}nyi divergence ($\alpha=-1$) \citep{li2016variational}
& $\le$ 140.5 \\
\bottomrule
\end{tabular}
\caption{Inference methods for a probabilistic decoder on binarized
MNIST. The Edward \gls{PPL} is a convenient research platform, making
it easy to both develop and experiment with many algorithms.}
\label{table:mnist}
\end{table}

We demonstrate Edward's flexibility for experimenting with complex
inference algorithms.
We consider the \gls{VAE} setup from \Cref{fig:vae} and the binarized
MNIST data set \citep{salakhutdinov2008quantitative}.  We use $d=50$
latent variables per data point and optimize using ADAM.  We study
different components of the \gls{VAE} setup using different methods;
\Cref{appendix:vae} is a complete script.  After training we evaluate
held-out log likelihoods, which are lower bounds on the true value.

\Cref{table:mnist} shows the results.  The first method uses the
\gls{VAE} from \Cref{fig:vae}. The next three methods use the same
\gls{VAE} but apply different gradient estimators: reparameterization
gradient without an analytic KL; reparameterization gradient with an
analytic entropy; and the score function gradient
\citep{paisely2012variational,ranganath:2014}.  This typically leads
to the same optima but at different convergence rates. The score
function gradient was slowest. Gradients with an analytic entropy
produced difficulties around convergence: we switched to stochastic
estimates of the entropy as it approached an optima. We also use
\glspl{HVM} \citep{ranganath2016hierarchical} with a normalizing flow
prior; it produced similar results as a normalizing flow on the latent
variable space \citep{rezende2015variational}, and better than
\glspl{IWAE} \citep{burda2016importance}.

We also study novel combinations, such as \glspl{HVM} with the
\acrshort{IWAE} objective, \gls{GAN}-based optimization on the
decoder (with pixel intensity-valued data),
and R\'{e}nyi divergence on the decoder.  \gls{GAN}-based
optimization does not enable calculation of the log-likelihood;
R\'{e}nyi divergence does not directly optimize for log-likelihood so
it does not perform well. The key point is that
Edward is a convenient research platform:
they are all easy modifications of a given script.


\subsection{GPU-accelerated Hamiltonian Monte Carlo}
\label{sub:gpu}

\begin{figure}[!htb]
\vspace{-1.5ex}
\begin{subfigure}{0.3\columnwidth}
  \centering
  \begin{tikzpicture}

  % Nodes
  \node[latent]              (beta)      {$\beta$} ;
  \node[right=0.4cm of beta]               (betaph)      {} ;
  \node[obs, below=0.75cm of betaph]      (y)    {$\mby_n$} ;
  \node[obs, left=1.0cm of y] (x)    {$\mbx_n$} ;

  % Edges
  \edge{beta}{y};
  \edge{x}{y};

  % Plates
  \plate[inner sep=0.25cm,
    label={[xshift=-15pt,yshift=15pt]south east:$N$}] {plate1} {
    (y)(x)
  } {};

\end{tikzpicture}

\end{subfigure}%
\begin{subfigure}{0.6\columnwidth}
  \centering
\begin{lstlisting}[language=python]
  # Model
  x = tf.Variable(x_data, trainable=False)
  beta = Normal(mu=tf.zeros(D), sigma=tf.ones(D))
  y = Bernoulli(logits=tf.dot(x, beta))

  # Inference
  qbeta = Empirical(params=tf.Variable(tf.zeros([T, D])))
  inference = ed.HMC({beta: qbeta}, data={y: y_data})
  inference.run(step_size=0.5 / N, n_steps=10)
\end{lstlisting}
\end{subfigure}
\caption{Edward program for Bayesian logistic regression with \gls{HMC}.}
\label{fig:logistic_regression}
\vspace{-1.5ex}
\end{figure}

\begin{table}[tb]
\centering
\begin{tabular}{lr}
\toprule
Probabilistic programming system & Runtime (s)
\\
\midrule
Handwritten NumPy (1 CPU) & 534 \\
Stan (1 CPU) \citep{carpenter2016stan} & 171 \\
PyMC3 (12 CPU) \citep{salvatier2015probabilistic} & 30.0 \\
\textbf{Edward (12 CPU)} & \textbf{8.2} \\
Handwritten TensorFlow (GPU) & 5.0 \\
\textbf{Edward (GPU)} & \textbf{4.9}\\
\bottomrule
\end{tabular}
\caption{\gls{HMC} benchmark for large-scale logistic regression.
Edward (GPU) is significantly faster than other systems. In addition,
Edward has no overhead: it is as fast as handwritten TensorFlow.}
\label{table:hmc}
\end{table}

We benchmark runtimes for
a fixed number of Hamiltonian Monte Carlo \citep[\gls{HMC};][]{neal2011mcmc}
iterations on modern
hardware: a 12-core Intel i7-5930K CPU
at 3.50GHz and an NVIDIA Titan X (Maxwell) GPU. We apply
logistic regression on the
Covertype dataset ($N=581012$, $D=54$; responses were
binarized)
using Edward,
Stan (with PyStan) \citep{carpenter2016stan}, and PyMC3
\citep{salvatier2015probabilistic}.
We ran 100 \gls{HMC} iterations,
with 10 leapfrog updates per iteration, a step size of $0.5 / N$, and
single precision.
\Cref{fig:logistic_regression} illustrates the program in Edward.

\Cref{table:hmc} displays the runtimes.%
\footnote{In a previous version of this paper, we reported PyMC3 took 361s. This was
caused by a bug preventing PyMC3 from correctly handling
single-precision floating point. (PyMC3 with double precision is roughly 14x slower than Edward (GPU).) This has been fixed
after discussion with Thomas Wiecki. The reported numbers also
exclude compilation time, which is significant for Stan.}
Edward (GPU) features a dramatic 35x speedup over Stan
(1 CPU) and 6x speedup over PyMC3 (12 CPU).
This showcases the value of building a \gls{PPL} on top of
computational graphs.
The speedup stems from fast matrix multiplication
when calculating the model's log-likelihood; GPUs can efficiently
parallelize this computation.
We expect similar speedups for models whose bottleneck is also matrix multiplication, such as deep neural networks.

There are various reasons for the speedup. Stan only used 1
CPU as it leverages multiple cores by running \gls{HMC} chains in
parallel. Stan also used double-precision floating point as it
does not allow single-precision. For PyMC3, we note Edward's speedup is
not a result of PyMC3's Theano backend compared to Edward's
TensorFlow. Rather, PyMC3 does not use Theano for all its
computation, so it experiences communication overhead with NumPy.
(PyMC3 was actually slower when using the GPU.)
We predict that porting Edward's design to Theano would feature
similar speedups.

In addition to these speedups, we highlight that Edward has no runtime
overhead: it is as fast as handwritten TensorFlow.  Following
\Cref{sub:inference}, this is because the computational graphs for
inference are in fact the same for Edward and the handwritten code.

\subsection{Probability Zoo}

In addition to Edward, we also release the \emph{Probability Zoo}, a
community repository for pre-trained probability models and their
posteriors.\footnote{%
  The Probability Zoo is available at \url{http://edwardlib.org/zoo}.
  It includes model parameters and inferred posterior factors, such as
  local and global variables during training and any inference
  networks.  } It is inspired by the model zoo in Caffe
\citep{jia2014caffe}, which provides many pre-trained discriminative
neural networks, and which has been key to making large-scale deep learning
more transparent and accessible. It is also inspired by Forest
\citep{stuhlmueller2012forest}, which provides examples of
probabilistic programs.
