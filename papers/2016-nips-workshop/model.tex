\section{Compositional Representations for Probabilistic Models}
\label{sec:modeling_language}

We define random variables as a key compositional representation.
They are class objects with methods, for example, to compute the log density
and to sample. Further, each random variable $\mbx$ is associated to a
tensor $\mbx^*$ in the computational graph, which represents a single
sample $\mbx^*\sim p(\mbx)$.  This association embeds the random
variable into the computational graph.

This design is conceptually simple, making it easy to develop
probabilistic programs in a computational graph framework.
Importantly, all computation is represented on the graph.  This makes
it easy to parameterize random variables with complex deterministic
structure, such as with deep neural networks and a diverse set of math
operations.  The design also enables compositions of random variables
to capture complex stochastic structure.

With computational graphs, it is also natural to build mutable states
within the probabilistic program.  As a typical use of computational
graphs, such states can define model parameters; in TensorFlow, this
is given by a \texttt{tf.Variable}.  Another use case is for building
discriminative models $p(\mby\g\mbx)$, where
$\mbx$ are features that are input as training or test data.  The
program can be written independent of the data, using a mutable state
(\texttt{tf.placeholder}) for $\mbx$ in its graph.
During training and testing, we feed the placeholder the appropriate
values.
In \Cref{appendix:bnn}, we demonstrate this with a Bayesian neural
network for classification.
We give other examples below.


\subsection{Example: Variational Auto-encoder}

\begin{figure}[tb]
\begin{subfigure}{0.225\columnwidth}
  \centering
  \begin{tikzpicture}

  % Nodes
  \node[latent] (z) {$\mbz_n$};
  \node[obs, below=of z] (x) {$\mbx_n$};

  \factor[empty, below=of z] {h} {} {} {};
  \factor[right=of h, xshift=0.5cm] {theta} {$\theta$} {} {};
  \factor[left=of h, xshift=-0.5cm] {phi} {$\phi$} {} {};

  % Edges
  \edge{z}{x};
  \draw[style=arrow, densely dotted, bend left] (x) to (z);
  \edge{theta}{x};
  \draw[style=arrow, densely dotted] (phi) to (z);

  % Plates
  \plate[inner sep=0.4cm, yshift=0.05cm,
    label={[xshift=-14pt,yshift=14pt]south east:$N$}] {plate1} {
    (z)(x)
  } {};

\end{tikzpicture}

  \label{sub:vae_math}
\end{subfigure}%
\begin{subfigure}{0.65\columnwidth}
  \centering
\begin{lstlisting}[language=python]
# Probabilistic model
z = Normal(mu=tf.zeros([N, d]), sigma=tf.ones([N, d]))
h = slim.fully_connected(z, 256)
x = Bernoulli(logits=slim.fully_connected(h, 28 * 28, activation_fn=None))

# Variational model
qx = tf.placeholder(tf.float32, [N, 28 * 28])
qh = slim.fully_connected(qx, 256)
qz = Normal(mu=slim.fully_connected(qh, d, activation_fn=None),
            sigma=slim.fully_connected(qh, d, activation_fn=tf.nn.softplus))
\end{lstlisting}
  \label{sub:vae_code}
\end{subfigure}
\caption{\Acrlong{VAE} for a data set of $28\times 28$ pixel images:
(left) graphical model, with dotted lines for the inference
model; (right) probabilistic program,
with 2-layer neural networks.
}
\label{fig:vae}
\end{figure}

\Cref{fig:vae}
implements
a \gls{VAE} \citep{kingma2014autoencoding,rezende2014stochastic} in Edward.
There are $N$ data points $\{x_n\}$ and $d$ latent
variables per data point $\{z_n\}$.
The program uses TensorFlow Slim \citep{guadarrama2016tensorflow}
to define the neural networks.
The probabilistic model is parameterized by a 2-layer neural network,
with 256 hidden units (and ReLU activation), and generates $28\times 28$
pixel images.  The variational model is parameterized by
a 2-layer inference network, with 256 hidden units and outputs
parameters of a normal posterior approximation.

The probabilistic program is concise.  Importantly, core elements of
the \gls{VAE}---such as its distributional assumptions and neural net
architectures---are all extensible.
Model compositionality enables it to be embedded into more complicated
models \citep{gregor2015draw,rezende2016one} and for other learning
tasks \citep{kingma2014semi}.
Inference compositionality (which we discuss later) enables it to be embedded into more
complicated algorithms, such as with expressive variational
approximations \citep{rezende2015variational,tran2016variational,kingma2016improving}
and alternative objectives \citep{ranganath2016operator,li2016variational,dieng2016chi}.


\subsection{Stochastic Control Flow and Model Parallelism}

\begin{figure}[!htb]
  \centering
  \begin{tikzpicture}[x=1.7cm,y=1.8cm,scale=0.9]
%\begin{tikzpicture}[scale=0.6]

  % Nodes
  \node[latent] (p) {$\mbp$};
  \factor[right=of p, xshift=0.3cm] {pstar} {$\mbp^*$} {} {};

  \factor[above=of pstar] {n} {} {} {};
  \factor[empty, right=of n, yshift=0.1cm] {nn} {\texttt{tf.while_loop(...)}} {} {};
  \factor[left=of n, xshift=-0.5cm] {astar} {$\mba^*$} {} {};
  \node[latent, left=of astar, xshift=0.5cm] (a) {$\mba$};
  \node[latent, right=of pstar, xshift=-0.5cm] (x) {$\mbx$};
  \factor[right=of x, xshift=0.3cm] {xstar} {$\mbx^*$} {} {};

  % Edges
  \edge{p}{pstar};
  \edge{pstar}{x};
  \edge{n}{x};
  \edge{x}{xstar};
  \edge{a}{astar};
  \edge{astar}{n};

\end{tikzpicture}

\caption{Computational graph for a probabilistic program with stochastic control flow.
}
\label{fig:dynamic}
\end{figure}

Random variables can also be integrated with control flow,
enabling probabilistic programs with stochastic control flow.
%
Stochastic control flow defines dynamic conditional dependencies,
known in the literature as contingent or existential dependencies
\citep{mansinghka2014venture,wu2016swift}.
See \Cref{fig:dynamic}, where $\mbx$ may or may not depend on $\mba$
for a given execution.

We use stochastic control flow to implement a Dirichlet process mixture model
in \Cref{appendix:dirichlet_process}.
Stochastic control flow produces difficulties for algorithms that leverage the
graph structure; the relationship of conditional dependencies
changes across execution traces.
Importantly, the computational graph provides
an elegant way of teasing out static conditional dependence structure
($\mbp$) from dynamic dependence structure ($\mba)$.  We can perform
model parallelism over the static structure with \glspl{GPU} and batch
training, and use generic computations to handle the dynamic
structure.
